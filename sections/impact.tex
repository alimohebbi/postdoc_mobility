\subsection{Relevance and impact of the research }


In the past decades, researchers focused on automated GUI testing as a suitable solution for reducing the cost of integration testing. However, the current solutions have not been wildly used by practitioners due to their limitations. 
Our approaches address the limitations and impact both research and industry as follows:


% on research 


\smallskip
\noindent
\textbf{Research:}

\noindent
This project produces mature \testreuse approaches, and it will impact the broader field of GUI testing by following contributions: 
\begin{inparaenum}[(i)]
\item proposes a set of prompt patterns that formulates \testreuse as a prompt answering problem 
\item proposes strategies for building comprehensive  \tam
\item enhances the effectiveness of semantic matching techniques.
\end{inparaenum}

\smallskip
The first two contributions impact model-based approaches from two aspects.
First, integration of GUI model in prompts opens a new direction to use LLM for model-based testing.
Second, a comprehensive model improves coverage of the  generated test cases.
Effective semantic matching influnce Test Repair and Pattern-based approaches, both of which rely on semantic matching.
Test repair approaches find replacements for events that are broken in a test case between two versions of an app by matching the broken event to a semantically equivalent event in the new version~\cite{Pan:Meter:TSE:2022}.
Pattern-based approaches use semantic matching to match elements of an abstract pattern to events of the target app.


\smallskip
\noindent
\textbf{Industry:}

\noindent
The approaches that we introduce in this project are applicable to all types of applications that include GUI interactions, such as E-commerce and mobile apps.
E-commerce sales exceeded 6 trillion U.S. dollars worldwide~\footnote{\href{https://www.statista.com/topics/871/online-shopping/}{E-commerce worldwide - statistics \& facts}} in 2023.
%These commercial sectors are just a subset of industries that depend on GUI applications, and the financial cost of software failure is substantial in them.
The financial cost of software failure is substantial in these sectors.
In this project, we will create a standalone tool, \rltool, that developers can use to reveal functional faults, improve software quality, and ultimately increase business revenues.
%
Additionally, we will introduce an effective semantic matching technique that is applicable not only in testing but also in business logic.
Many organizations analyze available data on the web to make business decisions or provide insights. 
Various sources may represent the same data differently, which decreases analysis accuracy. 
Developers can leverage semantic matching approaches to overcome this issue.
