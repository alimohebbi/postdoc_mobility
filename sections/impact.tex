\subsection{Relevance and impact of the research }

\ali{to be revised}

% section 6 of sam proposal
% 750 words
The approaches introduced in this project are applicable to all type of application that have GUI including E-commerc and mobile app markets.
Today number of E-commerce has the revenue of more than USD 6 trillion \footnote{https://www.statista.com/topics/871/online-shopping/} and 
The global mobile app development market size was valued at USD 362 billion in 2022\footnote{https://www.statista.com/outlook/dmo/app/worldwide}.
The financial cost of software failure is substantial in these sectors. 
Our approaches impact both research and industry as follows:


% on research 


\smallskip
\noindent
\textbf{Research:}

\noindent
This project impacts the research areas of test generation by introducing following contributions:
\begin{inparaenum}[(i)]
\item Proposes techniques for addressing widget captioning problem
\item Proposes set of pattern prompts that enhances formulating GUI testing  as prompt answering
\item proposes strategies for building comprehensive \tam
\end{inparaenum}
More comprehensive GUI models can be used to improve current model-based test generation approaches.
Advancement in semantic matching has a broader impact on software engineering domain, such as GUI Test Repair and Pattern-based test generation approaches that heavily rely on semantic matching .
Test repair approaches needs to find replacement for broken events in a test case between to versions of app by finding a proper match from the broken event in the previous version to a semantically equivalent event in the new version.
Pattern-based approaches rely on semantic matching to match element of an abstract pattern to actual events of the target application.
Additionally, inferring semantic of GUI by approaches that we propose leads to improving accessibility of applications.
The our approaches can infers semantics of GUI and the result would be translated to proper representation for users with visual impairment.  


\smallskip
\noindent
\textbf{Industry:}

\noindent
We increase applicability of the software by making a reliable and effective \testreuse tool, \toolreuse. 
We will finalize our approaches as a standalone tool that developer can easily use it and it reduces their effort.
As a result quality of applications will increase, while the cost of development decrease wider employment of automated testing tools.
Eventually, higher quality products increases businesses revenues.

Advanced semantic matching techniques in our \testreuse approaches, not only have applicability in software development life-cycle such as testing, but also can be part of business logic. 
Many business and organization uses available data on the web to make business decisions. 
However, various sources may represent the same data differently and that decreases accuracy of data analysis.
Advanced semantic matching approaches that we envies in this project can be leveraged to overcome this issue.
