	\section{Summary of the research plan}
	%1000 words
	% What should be included in here? summary of all below sections?


% What problem project aims to solve?
In this project we defined a set of approaches to automatically test application through their Graphical User Interface (GUI).
Our approaches generate test cases that exercise the application under the test with realistic scenarios and reveal functional faults.


% What is the gap in the research?
Testing applications is a crucial step in development process that increase quality of the application.
GUI testing tests applications at integration level to ensure their compliance with the functional requirements. 
Generating test cases manually, which is the common practice in industry, bear substantial cost on organizations and it is tedious.
Most of automated approaches only maximize code coverage and reveal faults that lead to system crashes.
However, they ignore functionality of applications and miss the functionality related faults.
Recently, researchers proposed approaches, such as \testreuse, that are aware of semantics of applications and they generate test cases that exercise the functionalities  similarly to users.


\testreuse is a promising category of automated GUI testing approaches that migrate test cases of an application to another with shared similar functionalities.
These approaches are recent and they have limitations, thus they require improvement to be fully applicable in industrial environments. 
The current \testreuse approaches are inaccurate in migrating test cases due to solely relying on textual information in the GUI for inferring semantics, and poor decision making strategies for dealing with lack of information.  


% What are the objectives? We do X to Y
In this project, we will address current limitations of \testreuse approaches by
\begin{inparaenum}[(i)]
\item using visual information of GUI to enrich internal information used in the test migration
\item using patterns of GUI interactions and web information to incorporate external information
\item defining strategies to build comprehensive models of GUI that facilitate navigation and decision making of \testreuse approaches.
\end{inparaenum}

% How we do that? We use technique W to do X

We will leverage Artificial Intelligence to define  and integrate set of \testreuse approaches together to create a  \textbf{H}ighly \textbf{In}telligent \textbf{Te}st \textbf{R}euse (\project) approach.
we will
\begin{inparaenum}[(i)]
\item use Computer Vision to extract visual information in the GUI
\item  use Large Language Models to integrate their vast knowledge and pattern recognition capabilities
\item use Reinforcement Learning to efficiently build models for  GUI of applications.
\end{inparaenum}
% How we validate the approach?
We will validate our approaches on the most popular applications from different domains to demonstrate their general applicability and  their effectiveness in industrial environment.


% What is the impact of the project?

This project will impact both research and industry. 
It will impact research in software engineering by introducing approaches that generate test case at integration level, and methods that benefits other testing areas, such as test repair. 
It will impact on development of software applications by increasing their quality and reducing production cost.




























