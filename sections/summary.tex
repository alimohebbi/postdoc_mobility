	\section{Summary of the research plan}
	%1000 words
	% What should be included in here? summary of all below sections?

% What problem project aims to solve?
In this project we define a set of approaches to generate test cases for testing software applications through their Graphical User Interface. 
The generated test case are able to exercise the application under the test with realistic scenarios and reveal functional faults.

\bigskip
% What is the gap in the research?
Testing software applications is crucial to increase quality of a software. 
GUI testing tests software application at integration level to ensure its compliance to the functional requirements. 
Usually developers write test cases manually, which is tedious task and bears substantial cost on organizations.
Researcher have proposed many approaches to automate generating GUI test cases. 
Most of these approaches only are able to maximize code coverage and reveal faults that lead to system crashes.
Recently, researcher have proposed approaches that are aware of  semantics of applications and  test  intended functionality of the applications.

\bigskip
\testreuse is a promising category of automated GUI testing testing approaches that migrates tests of an application to another application with similar functionalities. 
These approaches are far from prefect, which hinders their applicability in industrial environment.
The current \testreuse approaches only rely on limited amount of information in the GUI to infer semantics and that leads to imprecise test migration with high uncertainty.

\bigskip
% What are the objectives? We do X to Y
In this project, we will address current limitations of \testreuse approaches by
\begin{inparaenum}[(i)]
\item using visual information of GUI to incorporate internal information in test migrations
\item using pattern of GUI interaction and web information to incorporate external information
\item defining strategies to build comprehensive model of target application GUI that facilitates navigation of \testreuse approaches in the target applications
\end{inparaenum}.

% How we do that? We use technique W to do X
\bigskip
We will leverage Artificial Intelligence to define set of \testreuse approaches and integrate them together to create a \textbf{H}ighly \textbf{In}telligent \textbf{Te}st \textbf{R}euse (\project) approach.
we will
\begin{inparaenum}[(i)]
\item incorporate Image Processing to extract visual information in the GUI
\item use Large Language Models to integrate their vast knowledge and pattern recognition capabilities into \testreuse
\item Reinforcement Learning  to efficiently build a model of target application GUI
\end{inparaenum}

% How we validate the approach?




% What is the impact of the project?