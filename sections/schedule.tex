\subsection{Schedule and milestones}
% 1000 words

% \usepackage{rotating}
% \usepackage{multirow}


\begin{table}[h!]
	\centering
	\caption{Research Plan}
	\label{tab:research-plan}
	\begin{tabular}{|m{1cm}|m{1cm}|p{4cm}|p{4cm}|p{4cm}|} 
		\cline{2-5}
		\multicolumn{1}{c|}{\multirow{2}{*}{}}                      & \multirow{2}{*}{\textbf{Duration}}     & \multicolumn{3}{c|}{\textbf{Objectives}}                                                                                                                                                                                                                                                                                                                                                                                                                                                                                                                                                                                                                                                                                                                                     \\ 
		\cline{3-5}
		\multicolumn{1}{c|}{}                                       &                                        & \textbf{O1 (Internal Resources)}                                                                                                                                                                                                                                                             & \textbf{O2 (External Resource)}                                                                                                                                                                                                               & \textbf{O3 (Robustness)}                                                                                                                                                                                                                  \\ 
		\hline
		\rotatebox{90}{\textbf{\textbf{Study}}}      & \rotatebox{90}{2 months} & Studying computer vision approaches for classifications of images             Choosing most suitable approach in context of test reuse                                                                                                                                           & Studying LLM approaches             Identifying suitable prompts template                                                                                                                                                                     & Studying RL approaches             Identifying suitable RL approaches in Test Reuse context             Identifying different options for representing states and rewards                                                                 \\ 
		\hline
		\begin{sideways}\textbf{\textbf{Experiments}}\end{sideways} & \begin{sideways}4 months\end{sideways} & Collecting trainset and training a vision-based classifier             Integrating CraftDroid with the classifier             Experimenting on Subjects used by our previous study for generating test cases             Experimenting on new Subjects for generating test cases & Preliminary experiments with different LLM approaches and prompts and choose the best one             Fine-tuning LLM models             Integrating CraftDroid with the LLM models             Experimenting on all of subjects that we have & Preliminary experiments with different representation of states             Using RL in present of uncertainty             Using RL to explore GUI and make GUI model complete             Experimenting on all of subjects that we have  \\ 
		\hline
		\begin{sideways}\textbf{\textbf{Report}}\end{sideways}      & \begin{sideways}2 months\end{sideways} & Submitting a workshop paper from the preliminary studies             Submitting a research conference paper                                                                                                                                                                                                          & Submitting a workshop paper from the preliminary studies             Submitting a research conference paper                                                                                                                                   & Submitting a journal paper                                                                                                             \\
		\hline
	\end{tabular}
\end{table}

Table~\ref{tab:research-plan} summarizes the schedule and milestones of the \project project.
In our schedule, we achieve an objective by completing a milestone  consisting of three steps: Study, Experiments, and Report. 
 In the study phase, we will review the most recent works related to the objective of its milestone.
In the experiment phase, we will gather a complete data set for training models,  identify subjects, and empirically evaluate the new approach. 
In the report phase, we will submit the results of experimenting with our approaches in  the top research conferences and journals of Software Engineering.
We explain the details of milestones below.

\smallskip
\noindent
\textbf{O1: Incorporating Internal Resources:}  

\noindent
In the study phase, we will review vision-based GUI testing  and widget recognition approaches. 
We will examine the replication package and data sets of these studies, and in case of compatibility with our context, we will reuse them. 
Otherwise, we will study image captioning and classification approaches to build our model for the \imagelabeler component.
%\bigskip
%There is a risk that existing approaches for recognition of widgets will not be replicable. 
%In that case we need to take an alternative route which is classifier type, building training set, and training a classifier.  
%Therefore, we study computer vision approaches to identify most suitable classifier. 
%The most common approach for image classification is Convolutional Neural Networks (CNN)~\cite{lecun1995convolutional}. 
%Using CNN requires careful choice of its architecture depending on the down stream tasks. 
%We will investigate the most popular choices that includes: ResNet~\cite{he:ResNet:CVPR:2016}, Inception~\cite{szegedy:inception:CVPR:2015}, MobileNet~\cite{howard:mobilenets:arxiv:2017}.
%
%\bigskip
We will transition from the study to the experiment phase by conducting a feasibility study. 
We will investigate from suitable approaches that we identified which one is more promising in our context.
To do so, we will experiment with different prototypes of \imagelabeler for semantic matching in isolation from the \testreuse,  similar to our previous study~\cite{mariani:SemFinder:ISSTA:2021}. 
Then, we will integrate \imagelabeler in our \tme framework to create a new \testreuse approach named \visiontool\footnote{Horus is a god in ancient Egyptian mythology that symbolizes heightened vision.}.
Additionally, we will use our\tme framework to evaluate \visiontool and comparing with the state-of-the-art \testreuse approaches.
Finally, we will submit the results of our  feasibility study to a workshop conference and the result of the \testreuse experiments to a top software engineering conference. 

%%% O2


\smallskip
\noindent
\textbf{O2: Incorporating External Resources:}  

\noindent
In the study phase, we will review LLM-based GUI testing approaches~\cite{Zimmermann:GPT3GUITest:2023:ICSTW, liu:GUIInputLLM:ICSE:2023} and prompt engineering, as a recent study shows tuning prompts improve code summarization  by 26\% on average~\cite{wang:prompt:FSE:2023}.
%\bigskip
We will transition from the study phase to the experiment phase by conducting a feasibility study that examines different LLM models and prompt patterns.
We will investigate the applicability of existing LLM models in the literature in our context or if we need to create a new model by fine-tuning them.
We will examine different combinations of LLM models and prompt patterns to translate a limited number of test cases.
We will manually inspect their quality and choose the most promising combination.
After the feasibility study, we will integrate the \llmtranslator into our framework to create a new \testreuse approach named \llmtool\footnote{Hermes is a god in ancient Greek mythology that symbolizes linguistic prowess.}.
Additionally, we will use our \tme framework to evaluate \llmtool and compare with the state-of-the-art \testreuse approaches.
Finally, we will submit the results of our  feasibility study to a workshop conference and the result of the \testreuse experiments to a top software engineering conference. 


%%% O3


\smallskip
\noindent
\textbf{O3: Improving Robustness:}  

\noindent
%study
In the study phase, we will review Reinforcement Learning based  GUI testing approaches~\cite{Mariani:GUI:STVR:2014,Vuong:RLTest:A-Test:2018,Pan:QTesting:ISSTA:2020,Romdhana:ARES:TOSEM:2022}. 
We will identify suitable options for the RL approach, state presentation, and reward function.
%Experiment
We will transition from the study to the experiment phase by  conducting a feasibility study, in which we evaluate candidate \rlaganets performance based on the number of distinctive windows  explored in a given amount of time for a subset of our subjects. 
After the feasibility study, we will  integrate the chosen \rlaganet in our framework to create a new \testreuse approach named \rltool\footnote{Athena is a goddess in ancient Greek mythology that symbolizes strategic intellect.
}.
\rltool subsumes \visiontool and \llmtool by  encompassing \imagelabeler and \llmtranslator components.
Additionally, we will use our \tme framework to evaluate \rltool and compare it with the state-of-the-art \testreuse approaches.
%We will evaluate the effectiveness of \rlaganet in three setups:
%\begin{inparaenum}[a)]
%	\item The agent only executes before \testreuse to make GUI model more complete
%	\item The agent only executes when semantic matching is uncertain
%	\item The agent executes both in the beginning and uncertain conditions.
%\end{inparaenum}
Finally, we will submit the \rltool in a tool track of a conference and the result of the \testreuse experiments in a top software  engineering journal to conclude the project. 
















