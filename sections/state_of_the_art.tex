\subsection{Current state of research in the field }
%1000 words

\subsubsection{Terminology}
A Graphical User Interface is a forest of hierarchical windows, and only one window at a time is active to be used. Windows include atomic elements named widgets that are characterized by attributes such as text or location. At any given time, the active windows have a state that encompasses attribute values of all visible widgets. The type of widgets depends on their functionality, and some widgets expose user-actionable events. 

\smallskip
A GUI event is an atomic human-computer interaction, such as clicking on a widget with the type of button.
%An oracle event checks the state of a widget. 
%For example, if a widget displays a specific text. 
A GUI test case is a sequence of events $\langle e_1,..., e_n\rangle$ on widgets of active windows.
A test execution results transitioning the state of the active windows $S_{0} \xrightarrow{e_1} S_1 \xrightarrow{e_2} S_2 \ldots \xrightarrow{e_n} S_{n}$ 
%\), 
where $S_{i-1}$ and $S_i$ denote the states of the active window before and after the execution of $e_i$, respectively. 
A GUI model (\tam) is a directed graph where nodes correspond to GUI states, and edges correspond to events that transition the source node (state before the event) to the target node (state after the event). 

% Test Reuse

\smallskip
\testreuse is the process of migrating test cases across apps with similar functionalities. \testreuse relies on semantic matching to find corresponding events from the source test case in the target application. The semantic matching process scores the similarity of events using textual attributes of widgets.
The target test case may include stepping events that do not correspond to any events in the source test case but are required to reach relevant states. 
The source test cases may include unmatchable events that do not correspond to any event in the target test case. 
%\testreuse handles stepping and unmatchable events relying on the \tam. 




% Non semantic appraoches 

% Patter-based
% Test Reuse
% leveraging vision for test generation in random, pattern , ..
% Leveraging LLM for test generatoin,input, but no GUI


% What is LLM?

% All approaches
\subsubsection{State of the art}
Researchers have proposed many GUI test generation approaches that we classify as four main  categories: random-based, model-based, coverage-based, and similarity based approaches.
Random-based approaches generate test cases taking random actions to interact with the GUI~\cite{machiry:dynodroid:FSE:2013,vos:testar:ijismd:2015,ermuth:monkey:ISSTA:2016}. 
Model-based approaches create a model of GUI and and generate test cases to cover the model based on a coverage criteria~\cite{Nguyen:GUITAR:ASEJ:2014,Li:DroidiBot:ICSE-C:2017,Gu:PractivalTest:ICSE:2019,Choi:swift:OOPSLA:2013}. 
Coverage-based approaches generate test cases that maximize the code coverage\cite{Gross:exist:ISSTA:2012,mahmood:evodroid:FSE:2014,dong:TimaMachine:ICSE:2020,cheng:guicat:ASE:2016,Anand:conc:FSE:2012}.
Similarity-based approaches uses existing knowledge of testing a functionality  to generate test cases for applications with similar functionalities. 
All of these approaches but semantic-based are agnostic to functionality of the application under the test. 
Non-semantic approaches generate many test cases in  a short time with high coverage, but the test cases only reveal  crashing faults and they cannot reveal faults that related to functionality of the applications.

% Pattern-
\bigskip 
There are millions of applications that share similar functionalities. 
Semantic-based approaches take advantages of this opportunity that similar functionalities can be tested similarly and create test cases that are semantically relevant to the application under the test.
Semantic-based approaches has two sub-categories:
Pattern-based approaches and \testreuse approaches.
Pattern-based approaches consider abstract recurrent events and match the pattern elements to the actual events in the target application~\cite{Moreira:pattern:ISSRE:2013,Morgado:Impact:HCI:2019}. 
They either consider a predefined set of patterns\cite{Mariani:Augusto:ICSE:2018,Hu:appflow:FSE:2018}, or automatically extract patterns from traces of user interactions\cite{linares:mining:MSR:2015,mao:crowd:ASE:2017,Mao:UserPattern:JSS:2021}.
Crafting patterns manually requires substantial human effort, and extracting the GUI interaction patterns requires huge data set.

\bigskip
\testreuse approaches leverage textual similarity of GUI elements to migrate test cases of a source application to a target application that shares the same functionality. 
Researchers proposed approaches to migrate test cases of the same  application across different platforms. 
TestMig reuse test cases of an application from IOS version to Android\cite{Qin:testmig:ISSTA:2019}.
MAPIT reuse test cases bidirectionally between IOS and Android version\cite{talebipour:MAPIT:ASE:2021}.
TransDroid reuses test cases of web apps to their Android equivalent\cite{lin:TransDroid:ICST:2022}.
Other approaches migrate test case of different application across the same platform. 
 Rau et al. proposed to migrate test cases in web applications by using semantic matching of GUI elements~\cite{rau:efficent:icse:2018}.
 \craftdroid migrates test cases across Android apps~\cite{lin:craftdroid:ASE:2019}.
 GUITestMigrator migrates test cases between prototype apps that share same specifications~\cite{Behrang:migra:ISSTA:2018}.
 \atm is the extension of GUITestMigrator that migrates test cases of real apps~\cite{Behrang:migration:ICSE:2018}.
 \adaptdroid formulates \testreuse as a search problem and uses an evolutionary algorithm to explore the space of possible GUI test cases~\cite{Mariani:Adaptdroid:AST:2021}.
 
 
\bigskip
Researchers leveraged Machine Learning approaches to create enhanced approaches in the all categories of GUI test generation approaches. 
%Vision
APPFLOW uses computer vision to map elements of predefined patterns to the element in the GUI of target application~\cite{Hu:appflow:FSE:2018}.
AVGUST leverages computer vision to extract interaction patterns from screen recording of users interacting with many apps~\cite{zhao:Avgust:FSE:2022} .
White et al. proposed an approach that improves random GUI testing by widget detection~\cite{white:WidgetDetection:ISSTA:2019}. 
%Their approach determines the widget type, such as the menu or text box. However, it does not provide information about the semantics of widgets.
Zhu et al. proposed an approach for identifying the intent of widgets and labeling them to describe the widget function~\cite{zhu2021widgetrecog}. 
%However, their approach is not examined in the GUI testing context. 
% RL
DeepGUI uses leverages Deep Reinforcement Learning  and Computer Vision to identify actionable event and select them randomly~\cite{YazdaniBanafsheDaragh:DEEPGUI:ASE:2021}.
Many approaches uses RL techniques such as Q-Learning to model behavior of the GUI.
AutoBlackTest generates test cases of desktop applications~\cite{Mariani:ABT:STVR:2014} and A-Test  targets Android apps~\cite{Vuong:RLTest:A-Test:2018}.
%Q-Testing abstract GUI states at higher granularity of functional scenarios. 
ARES uses deap learning to infer states similarities while other techniques use Tabular Reinforcement Learning~\cite{Romdhana:ARES:TOSEM:2022}.
Recently Large Language Models (LLM) have been used to generate inputs for editable widget in random GUI testing~\cite{liu:GUIInputLLM:ICSE:2023} and transforming natural language commands into GUI test cases\cite{Zimmermann:GPT3GUITest:2023:ICSTW}. 
 


% limitations of current test reuse approaches

\bigskip
\testreuse approaches are the most recent category of GUI test generation approaches. 
They provide a great opportunity to create realistic test cases that can expose functional faults.
However, current approaches have limitations that hinders their ability. 
Our review of state of the art and our studies suggests we can address these limitations by leveraging advanced machine learning techniques.
First limitation is that they only rely on textual information in the GUI to semantically matches events of two applications. 
In this project we plan to leverage computer vision to address this limitation by combining visual and textual information.
Additionally, we will use LLM models to incorporate external information into \testreuse. 
Second limitation is current approaches navigate in the target application by relying \tam that often is incomplete. 
In this project we plan to use Reinforcement Learning to create a more comprehensive \tam.



\subsubsection{Current Research Projects}



%Use mauro proposal

% Sam approache propose a general faremwork targets transfering test cases and complement current test reuse approaches by intergrating other category of test case generation such as model-based testing. While this project aims to address limitation of current test reuse approaches by leveraging artificial intelligence.