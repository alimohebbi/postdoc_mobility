\subsection{Current state of research in the field }

\subsubsection{Terminology}

%%% Background knowledge
% What is GUI test case
% What is current state
% What is GUI model
\ali{to be revised}

A Graphical User Interface is a  forest hierarchical windows that only one of them at a time is active to be used.  
Windows includes atomic element named widget that are characterize by attribute, such as text or locations.
At any time the active window has a sate that encompasses attribute values of all visible widgets.
Type of widget depends on their functionality and some of them expose user-actionable events for users to interact with the application.

A GUI event is an atomic humane-computer interaction, such as click on a widget with the type button or fill an editable widget.
An oracle event check the state of a widget. 
For example, if a widget displays a specific text. 
A GUI test is a sequence of events $\langle e_1,..., e_n\rangle$ on widgets of active windows.
A test execution results transitioning the state of the active windows $S_{0} \xrightarrow{e_1} S_1 \xrightarrow{e_2} S_2 \ldots \xrightarrow{e_n} S_{n}$ 
%\), 
where $S_{i-1}$ and $S_i$ denote the states of the active window before and after the execution of $e_i$, respectively. 
A GUI model (\tam) is a directed graph where node correspond to GUI states  and edges correspond to events that transition the the source node (state before event) to the target node (state after the event).
% Test Reuse
% Semantic matching is one-to-one and using GUI model alleviate this issue.
% What is stepping events

\testreuse is the process of migrating test cases across apps with similar functionalities. 
\testreuse relies on semantic matching to find a corresponding event from the source test case in the target applications (one-to-one matching). 
Semantic matching uses textual attributes of widgets to score similarity of events.
The target test case may include stepping events, that are event not corresponding to any event in source test case.
The stepping event are required to reach relevant states.
The source test case may include  unmatchable events, that are event not corresponding to any event in target test case.
\testreuse handles stepping and unmatchable by relying on \tam.




\subsubsection{State of the art}
GUI testing has been a popular field in software engineering over the past two decades. Researchers proposed many approaches that we classify into four categories: random-based, model-based, coverage-based, and similarity-based.
Random-based approaches take random actions to interact with the GUI~\cite{machiry:dynodroid:FSE:2013,vos:testar:ijismd:2015,ermuth:monkey:ISSTA:2016}.
Model-based approaches build a GUI model and generate test cases that cover the model based on coverage criteria~\cite{Nguyen:GUITAR:ASEJ:2014,Li:DroidiBot:ICSE-C:2017,Gu:PractivalTest:ICSE:2019,Choi:swift:OOPSLA:2013}. 
Coverage-based approaches employ search-based algorithms~\cite{Gross:exist:ISSTA:2012,mahmood:evodroid:FSE:2014,dong:TimaMachine:ICSE:2020} or symbolic execution to generate test cases that maximize code coverage~\cite{Ganov:GUIsymbolic:FMSE:2009,cheng:guicat:ASE:2016,Anand:conc:FSE:2012}.
Similarity-based approaches use existing knowledge of testing functionalities of applications to generate test cases for applications with similar functionalities.
All the categories except for similarity-based are agnostic to the semantics of the application under the test. 
Non-semantic approaches quickly generate many test cases with high coverage, but they only reveal crashing faults and cannot expose functional faults.

% Pattern-
\smallskip 
Millions of applications share similar functionalities that create an opportunity to test them similarly.
For example, online booking of flights can be tested similarly across different applications. 
Similarity-based approaches leverage this opportunity and generate test cases relevant to the functionality of the applications.
We classify similarity-based approaches into pattern-based and \testreuse.
Pattern-based approaches consider patterns comprising abstract recurrent events and match elements of the pattern to actual events in the target application.~\cite{Moreira:pattern:ISSRE:2013,Morgado:Impact:HCI:2019}.
They either consider a predefined set of patterns~\cite{Mariani:Augusto:ICSE:2018,Hu:appflow:FSE:2018} or automatically extract patterns from traces of user interactions~\cite{linares:mining:MSR:2015,mao:crowd:ASE:2017,Mao:UserPattern:JSS:2021}.
Crafting patterns manually requires substantial human effort, while extracting patterns automatically requires enormous data.


\smallskip 
\testreuse approaches leverage semantic similarity of textual information available in the GUI to migrate test cases of an application (source) to another (target).
Researchers proposed approaches that migrate test cases of the same application across different platforms and different applications in the same platform. 
TestMig migrates test cases from the IOS version of an app to its Android version~\cite{Qin:testmig:ISSTA:2019}.
MAPIT migrates test cases bidirectionally between IOS and Android versions of the same app~\cite{talebipour:MAPIT:ASE:2021}.
TransDroid migrates test cases of web apps to their Android version~\cite{lin:TransDroid:ICST:2022}.
 Rau et al. proposed migrating test cases in web applications  using semantic matching of GUI elements~\cite{rau:efficent:icse:2018}.
 \craftdroid migrates test cases across Android apps~\cite{lin:craftdroid:ASE:2019}.
GUITestMigrator migrates test cases between prototype apps that share the same specifications.
\atm is an extension of GUITestMigrator that migrates test cases of real apps~\cite{Behrang:migration:ICSE:2018}.
 \adaptdroid formulates \testreuse as a search problem and uses an evolutionary algorithm to explore the space of possible GUI test cases~\cite{Mariani:Adaptdroid:AST:2021}.
 
 
\smallskip 
Researchers exploited Machine Learning techniques to propose enhanced approaches in different categories of GUI test generation.
%Vision
APPFLOW uses computer vision to match elements of predefined patterns to the GUI elements of the target application~\cite{Hu:appflow:FSE:2018}.
AVGUST leverages computer vision to extract patterns from screen recordings of users interacting with apps~\cite{zhao:Avgust:FSE:2022}.
White et al. proposed a random approach that detects the type of widgets from their visual appearance~\cite{white:WidgetDetection:ISSTA:2019}.
%%%%Their approach determines the widget type, such as the menu or text box. However, it does not provide information about the semantics of widgets.
Zhu et al. proposed an approach for identifying the intent of widgets~\cite{zhu2021widgetrecog}.
%%%%However, their approach is not examined in the GUI testing context. 
% RL
%%%%%%%%
DeepGUI uses deep reinforcement learning (RL) and computer vision to identify and select actionable events~\cite{YazdaniBanafsheDaragh:DEEPGUI:ASE:2021}.
Some approaches used reinforcement learning to model the behavior of the GUI and select events.
AutoBlackTests~\cite{Mariani:ABT:STVR:2014} and A-test~\cite{Vuong:RLTest:A-Test:2018} use Q-Learning to generates test cases  for desktop and Andoroid apps, respectively.
ARES uses deep learning to infer state similarities, while previous approaches used tabular reinforcement learning~\cite{Romdhana:ARES:TOSEM:2022}.
Recently, researchers used Large Language Models (LLM) to generate inputs for editable widgets in random GUI testing ~\cite{liu:GUIInputLLM:ICSE:2023} and to transform natural language commands into GUI test cases~\cite{Zimmermann:GPT3GUITest:2023:ICSTW}. 

% limitations of current test reuse approaches

\smallskip 
\testreuse approaches are the most recent category of GUI test generations, which can expose functional faults by exercising realistic scenarios. Our review of the state of the art and our empirical studies of \testreuse~\cite{mariani:SemFinder:ISSTA:2021,khalili:DomainEmbedding:ICPC:2022} suggest we can address the limitations of current approaches by using advanced machine learning techniques. 
The first limitation is that the current \testreuse approaches ignore visual information, and they only rely on text, which may not be sufficient. 
In this project, we will leverage computer vision to address this limitation by combining visual and textual information.
Additionally, we will use LLM models to incorporate web information that is integrated into their train set to improve \testreuse. 
The second limitation is that current approaches explore the target application by relying on \tam, which is often incomplete.
In contrast to model-based approaches that replaced the \tam with an RL agent, we will use RL to create a more comprehensive model.


\subsubsection{Current Research Projects}

Software testing is an important topic that is the subject of many research projects. The most relevant research projects that the applicant is aware of are the ERC and SNF projects. \textit{Testing the Untestable} (P.I. Lionel Briand) and \textit{Self-assessment Oracles for Anticipatory Testing} (P.I. Paolo Tonella) focus on the problem of testing machine learning, while this project uses machine learning to test GUI applications. \textit{A-Test Autonomic Testing} (P.I. Mauro Pezze) focuses on testing distributed systems in the production environment and uses machine learning for different types of applications and environments from this project. The first two projects are funded by ERC, and the last project is funded by SNF. 


